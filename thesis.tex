%!TEX spellcheck = en-AU
%
%% ----------------------------------------------------------------
%% thesis.tex -- MAIN FILE (the one that you compile with LaTeX)
%% ----------------------------------------------------------------

% Set up the document
\documentclass[a4paper, 11pt]{template/thesis} % Use the "thesis" style, based on the ECS thesis style by Steve Gunn
\graphicspath{{figures/}} % Location of the graphics files (set up for graphics to be in PDF format)
% You can also include multiple paths. eg: \graphicspath{{figures/}{figures/chapter1}{figures/chapter2}}

%\pdfminorversion=7
% Include any extra LaTeX packages required
\usepackage{amsmath}
\usepackage{amssymb} 
\usepackage{amsfonts}
\usepackage{tabularx}
\usepackage{enumitem}
\usepackage{xurl}
\usepackage{xeCJK}
\setCJKmainfont{Noto Serif CJK KR}

%\usepackage[pdftex]{graphicx}
\DeclareGraphicsExtensions{.pdf, .png, .jpg}
\usepackage[caption=false]{subfig}
\usepackage{subfiles}
\usepackage{algorithm}
\usepackage[noend]{algpseudocode}

\DeclareMathOperator*{\argmax}{arg\,max}
\DeclareMathOperator*{\argmin}{arg\,min}
\DeclareMathOperator*{\E}{\mathbb{E}}
\DeclareMathOperator{\PP}{\mathcal{P}}

% Programming environments: e.g. C++, Matlab, Python, ROS
\newcommand{\environment}[1]{\textsc{#1}}
% Code references: e.g. functions, variables, language keywords
\newcommand{\code}[1]{\texttt{#1}}
\newcommand{\edit}[1]{#1}
\usepackage{amsthm} 


\theoremstyle{plain}
\newtheorem{theorem}{Theorem}[chapter]
\newtheorem{corollary}[theorem]{Corollary}
\newtheorem{lemma}[theorem]{Lemma}
\newtheorem{proposition}[theorem]{Proposition}

\theoremstyle{definition}
\newtheorem{example}{Example}[chapter]
\newtheorem{assumption}{Assumption}[chapter]
\newtheorem{definition}{Definition}[chapter]
\newtheorem{problem}{Problem}[chapter]

\theoremstyle{remark}
\newtheorem{remark}{Remark}[chapter]

% \newtheorem{theorem}{Theorem}
% % \theoremstyle{definition}
% \newtheorem{example}{Example}
% \newtheorem{lemma}{Lemma}
% \newtheorem{remark}{Remark}
% \newtheorem{definition}{Definition}

%\hypersetup{urlcolor=blue, colorlinks=false} % Colours hyperlinks in blue, but this can be distracting if there are many links.

% Thesis Preamble 
\thesistitle {Prospection for Mobile Robots in Unknown Environments}
\keywordsvar {Mobile robots, Decision making under uncertainty, Information theory, Gaussian process}
\authors     {Ki Myung Brian Lee}
\supervisor  {Robert Fitch}
\cosupervisor  {Shoudong Huang, Chanyeol Yoo}

%% ----------------------------------------------------------------
\begin{document}
\frontmatter   % Begin Roman style (i, ii, iii, iv...) page numbering

% Set up the Title Page

\title {\thesistitlename}


%\authors {\texorpdfstring
%            {\href{Author@domain.com}{Author Name}}
%            {Author Name}
%            }
\addresses  {\groupname\\\deptname\\\univname}  % Do not change this here, instead these must be set in the "thesis.cls" file, please look through it instead
\date       {\today}
\subject    {}
\keywords   {\keywordsname}

\maketitle
%% ----------------------------------------------------------------

\setstretch{1.5}  % It is better to have smaller font and larger line spacing than the other way round


% ----------------------------------------------------------------
 %Declaration Page required for the thesis, your institution may give you a different text to place here
\Declaration{

\addtocontents{toc}{\vspace{1em}}  % Add a gap in the Contents, for aesthetics
\vspace{25pt}


I, Ki Myung Brian Lee, declare that this thesis, is submitted in fulfilment of the requirements for the award of Doctor of Philosophy, in the Faculty of Engineering and Information Technology at the University of Technology Sydney. 

This thesis is wholly my own work unless otherwise referenced or acknowledged. In addition, I certify that all information sources and literature used are indicated in this thesis. 

This document has not been submitted for qualifications at any other academic institution.

This research is supported by the Australian Government Research Training Program. 

\vspace{100pt}

Signed:\\
\rule[1em]{25em}{0.5pt}  % This prints a line for the signature

Date:\\
\rule[1em]{25em}{0.5pt}  % This prints a line to write the date
}
\cleardoublepage  % Declaration ended, now start a new page

%% ----------------------------------------------------------------
\fancyhead[RE,LO]{\it Abstract}
% The Abstract Page
\addtotoc{Abstract} % Add the "Abstract" page entry to the Contents
\abstract{
\addtocontents{toc}{\vspace{1em}} % Add a gap in the Contents, for aesthetics
%Suggested by Bruce as para for each: problem(s), method(s), results, conclusion

}

\cleardoublepage % Abstract ended, start a new page
%% ----------------------------------------------------------------
\fancyhead[RE,LO]{\it\leftmark}
\setstretch{1.5} % Reset the line-spacing to 1.3 for body text (if it has changed)

% The Acknowledgements page, for thanking everyone
\acknowledgements{
\addtocontents{toc}{\vspace{1em}} % Add a gap in the Contents, for aesthetics
}
\cleardoublepage % End of the Acknowledgements
% ----------------------------------------------------------------

%\input{./Chapters/Foreword} % Foreword
%\clearpage % End of the Acknowledgements
%\pagestyle{fancy} %The page style headers have been "empty" all this time, now use the "fancy" headers as defined before to bring them back

\fancyhead[RE,LO]{\it\leftmark}
%% ----------------------------------------------------------------

\tableofcontents % Write out the Table of Contents

% ----------------------------------------------------------------

\listoffigures % Write out the List of Figures

%% ----------------------------------------------------------------
\listoftables % Write out the List of Tables
%
%%% ----------------------------------------------------------------
\setstretch{1.5} % Set the line spacing to 1.5, this makes the following tables easier to read
\clearpage % Start a new page

%\input{chapters/acronyms}  % This file contains all the Acronyms, Abbreveations and Nomenclature.

\begingroup
\cleardoublepage % Start a new page

\renewcommand\bibname{List of Publications}
%\input{chapters/pub_list.bbl}
% Before uncommenting this build, chapters/pub_list.tex
\endgroup

%% ----------------------------------------------------------------
\mainmatter % Begin normal, numeric (1,2,3...) page numbering

\setstretch{1.5} % Reset the line-spacing to 1.3 for body text (if it has changed)

% To add: cognitive science
% Chapter 1
\chapter{Introduction}
\label{Chapter1}
\lhead{Chapter 1. \emph{Introduction}}

\lipsum[3-10]

%% Introduction Paragraph

%%%%%%%%%%%%%%%%%%
\section{Background}
\label{Chapter1:Background}


%%%%%%%%%%%%%%%%%%
\section{Motivation}
\label{Chapter1:Motivation}


%%%%%%%%%%%%%%%%%%
\section{Scope}
\label{Chapter1:Scope}


%%%%%%%%%%%%%%%%%%
\section{Contributions}
\label{Chapter1:Contributions}


%%%%%%%%%%%%%%%%%%
\section{Publications}
\label{Chapter1:Publications}

\subsection{Directly Related Publications}

\subsection{Related Publications}


%%%%%%%%%%%%%%%%%%
\section{Thesis Outline}
\label{Chapter1:ThesisOutline}



\todo{TODO}
 \clearpage % Introduction

% To add: bandit problems, MOMDP
%!TEX root = ../thesis.tex
%
% Chapter 2
\chapter{Review of Related Work}
\label{Chapter2}
\lhead{Chapter 2. \emph{Review of Related Work}}

\lipsum[11-15]

\todo[color=green]{TODO}
 \clearpage % Review of Related Work

%% ----------------------------------------------------------------
% Now begin the Appendices, including them as separate files

\addtocontents{toc}{\vspace{2em}} % Add a gap in the Contents, for aesthetics

\addtotoc{Appendices}

\appendix % Cue to tell LaTeX that the following 'chapters' are Appendices

% DISR workshop paper

% BOM technical report

%\input{./Appendices/AppendixA} \clearpage % Data Tables

\addtocontents{toc}{\vspace{2em}} % Add a gap in the Contents, for aesthetics
\backmatter

%% ----------------------------------------------------------------
\label{Bibliography}

\fancyhead[RE,LO]{\emph{Bibliography}} % Change the left side page header to "Bibliography"
\bibliographystyle{template/IEEEtran} % Use the IEEETran style
%\bibliographystyle{plain}
\bibliography{library} % The references (bibliography) information are stored in the file named "Bibliography.bib"


%compile todo list as well ---------------------------------
%\input{./todo} %To do list


\end{document} % The End
%% ----------------------------------------------------------------
